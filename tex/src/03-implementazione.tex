\section{Implementazione}
\subsection{Filtro FFT}
Nella pratica, il design presentato nella sezione precedente si rivela problematico.
Osserviamo lo spettrogramma di due segnali a 440 Hz con lo stesso contenuto in frequenza ma dove il secondo segnale presenta un numero di periodi non intero.

\begin{gnuplot}[terminal=epslatex, terminaloptions={color size 12cm,14cm}]
	set lmargin at screen 0.1
	set multiplot layout 2, 1
	
	set style line 11 lc rgb '#808080' lt 1
	set border 3 back ls 11
	set tics nomirror
	
	set style line 12 lc rgb '#808080' lt 0 lw 1
	set grid back ls 12
	
	set style line 1 lc rgb '#8b1a0e' pt 1 ps 1 lt 1 lw 2
	set style line 2 lc rgb '#5e9c36' pt 6 ps 1 lt 1 lw 2

	set title "Segnale d'esempio a 440 Hz; campionato a 44 kHz"
	plot "cosine.csv" using 1:2 w l ls 1 notitle
	set title "FFT"
	set logscale x 10
	set yrange [-200:0]
	plot "fft-int.csv" u 1:2 t "Segnale periodico" w l ls 1, "fft-nint.csv" u 1:2 t "Segnale non periodico" w l ls 2

#	set origin 0.15,0.05
#	set size 0.6,0.3
#	set xrange [330:660]
#	unset xlabel
#	unset ylabel
#	unset label
#	unset key
#	unset title
#	set xtics 110 offset 0,0.2
#	set ytics 200 offset 0.2,0.2
#	plot 'fft-int.csv' u 1:2 w lines ls 1, "fft-nint.csv" u 1:2 t "Segnale non periodico" w l ls 2

	unset multiplot
\end{gnuplot}

Nonostante il segnale sia lo stesso, uno dei due grafici presenta picchi di larghezza maggiore rispetto all'altro;
questo fenomeno, noto come ``spectral leakage'', produce informazioni imprecise riguardo il segnale analizzato, in particolare rispetto alle frequenze che si trovano ``in mezzo'' a due bin.
Ne consegue che azzerare il valore di un bin non ha effetto soltanto sulla frequenza desiderata ma comporta distorsione, producendo come risultato un'onda simile a un battimento piuttosto che silenzio.

Come soluzione è sufficiente applicare la FFT a un numero quanto più alto possibile di campioni.
Dato che il filtro deve essere applicabile a segnali audio arbitrari,
è stato scelto di aumentare il numero di campioni processando segmenti del file individualmente e aggiungere a questi un numero arbitrario di valori nulli.

La FFT calcola i coefficienti per un segnale periodico ma i segnali di nostro interesse sono limitati e raramente periodici.
Applicare semplicemente la FFT a un segmento di un segnale avvolge essenzialmente la fine del segmento all'inizio, generando una discontinuità di salto.
Tali salti si traducono in artefatti indesiderabili: per ridurli è possibile azzerare il segnale a entrambe le estremità del segmento moltiplicandolo con una funzione finestra.
Prima di appendere gli zeri viene quindi utilizzata la seguente funzione finestra
$$
w(n) = \frac{25}{46} - \frac{21}{46}\cos{\left( \dfrac{2\pi n}{N} \right)}, \ 0 \leq n \leq N
$$
nota come ``finestra di Hamming''\cite{Harris78onthe}.

I segmenti devono essere riuniti dopo essere processati individualmente per poter formare un nuovo segnale.
L'attenuazione agli estremi produce però un segnale con ampiezza non uniforme. Il problema viene risolto con overlay-add: si scelgono segmenti da $N$ campioni in modo che ogni segmento
si sovrapponga con il successivo per $\frac{N}{2}$ campioni, effettuando FFT a incrementi di $\frac{N}{2}$.
Le parti sovrapposte nel segnale in output vengono sommate e per simmetria della funzione finestra il segnale risultante avrà ampiezza uniforme.

È stato scelto di azzerare, oltre al bin corrispondente alla frequenza desiderata, un piccolo numero di bin adiacenti scelto sulla base della frequenza di campionamento.
La genericità del metodo lo rende appropriato per implementare filtri di diverso tipo in tempo reale.

\subsection{Filtro con trasformata zeta}
Per ottenere risultati migliori di quanto presentato nella sezione precedente si può far ricorso a un filtro Butterworth.
La progettazione di tale filtro può essere velocizzata utlizzando generatori, che hanno come in input solamente caratteristiche generiche
(ordine, frequenza di campionamento, tipo di filtro) e restituiscono i coefficienti del polinomio zeta.\\
L'implementazione risulta molto più semplice. Ad esempio, nel caso di un filtro al second'ordine, sia $x$ il segnale in ingresso composto da $N$ campioni, $z$ il segnale in uscita, $a_i$ l'i-esimo coefficiente del filtro
\begin{align*}
z_0 &= x_0\\
z_1 &= x_1\\
z_{n \ \in \ \{2, \ \dots, \ N\}} &= a_0x_n + a_1x_{n-1} + a_2x_{n-2} - a_3z_{n-1} - a_4z_{n-2} 
\end{align*}
permette di filtrare il segnale in $\Theta(N)$. Inoltre, non si pone il problema dello spectral leakage e di conseguenza la necessità di applicare padding e funzioni finestra.
D'altrocanto calcolare i coefficienti non è un'operazione computazionalmente leggera, inadatta per produrre filtri in tempo reale.
A questo si aggiunge il problema della frequenza di campionamento: è un'informazione essenziale nella progettazione di questo tipo di filtro, ed è necessario che coincida con quella del segnale da filtrare.
Si può risolvere il problema con un algoritmo di resampling, che interpola il segnale per far combaciare la frequenza di campionamento con quella del filtro, oppure precalcolando i coefficienti per le frequenze di campionamento più comuni.\\
In questa implementazione è stato scelto il secondo metodo, per mantenere la complessità sopracitata. Si noti che, a costo di un'implementazione leggermente più complicata, è possibile svolgere l'intera operazione in $O(1)$ spazio.

